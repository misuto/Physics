\documentclass[a4paper,11pt]{article}
\usepackage[T1]{fontenc}
\usepackage[utf8]{inputenc}
\usepackage{lmodern}
\usepackage{amssymb}
\title{Fysik}
\author{Jakob Tigerström}

\begin{document}
\tableofcontents
\newpage
\begin{flushleft}
\section{Metern}
Från början var en metern definerad av distansen mellan Nordpolen och ekvatorn so man bestämde var $ 10^7 $ meter.
Man gjorde kopior på metern som kallas arkivmetern.
1 meter är den sträcka som ljuset rör sig i vakum på $ \frac{1}{299792458} $ sekund.
\newline
\newline
\section{Massaenhet}
Kilogram
\newline
Arkivkilogram
\newline
\newline
\section{Tidsenhet}
Ursprungligen var sekunden $ \frac{1}{24*60*60} $ del av medelsoldygnet.
\newline
Idag är ett visst antal perioder av en viss strålning.
\newline
\newline
\section{SI-Systemet}
Bygger på att man har sju stycken noggrant definerade enheter. Som man sedan kan basera andra enheter på.
\newline
Härledda enheter:\newline
Areaenheter: $ m^2 $\newline
Volymenheter: $ m^3 $\newline
Hastighet:  \textit{m/s}\newline

Ex1: Vid en olje tanks rensning spreds 340 $ dm^3 $ olja ut på ett tunnt skikt på vattenytan.
Oljeskiktet var 2.5nm tjockt.\newline
Hur stor area hade oljebältet.
\newline
\begin{tabular}{l l | c r}
  Storhet & Beteckning & Enhet & Beteckning\\
  Längd & l & meter & m\\
  Massa & m & kilogram & kg\\
  Tid & t & sekund & s\\
\end{tabular}
\newpage
\section{Massa/volym}

\begin{tabular}{l c r}
  Massa(g) & Volym i mätglaset(ml) & Stenarnas volym(ml)\\
  0 & 62 & 0\\
  16.6 & 68 & 6\\
  29.9 & 73 & 11\\
  46.2 & 79 & 17\\
  62.9 & 85 & 23\\
  73.3 & 88 & 26\\
\end{tabular}
\newline
\newline
$ m = \rho * V  $
\newline
$ \rho = \frac{m}{V} $
\newline
$ \rho = 2.714285714 = \frac{76}{28} $\newline
$ \rho = 2,7 g/ml = \frac{2,6 g}{1 ml} = \frac{2,6 g}{0,001 dm} $
\newpage
\section{Prefix}

\begin{tabular}{l c r}
  femto & f & $ 10^{-15} $\\ 
  piko & p & $ 10^{-12} $\\ 
  nano & n & $ 10^{-9} $\\ 
  mickro & $ \mu $ & $ 10^{-6} $\\ 
  milli & m & $ 0,001 = 10^{-3} $\\
  centi & c & $ 0,01 = 10^{-2} $\\
  deci & d & $ 0,1 = 10^{-1}$\\
  Deka & da & $ 10 = 10^1 $\\
  hekto & h & $ 100 = 10^2 $\\
  kilo & k & $ 1000 = 10^3 $\\
  Mega & M & $ 10^6 $\\
  Giga & G & $ 10^9 $\\
  Tera & T & $ 10^{12} $\\
  Peta & P & $ 10^{15} $\\
  Exa & E & $ 10^{18} $\\
  Zetta & Z & $ 10^{21} $\\
  Yotta & Y & $ 10^{24} $\\
\end{tabular}
\newpage
EX1:\newline
En kula med radien 12,5 mm har massan 61g.\newline
Bestäm kulans densitet.\newline
$ m = 61g = 0,061 kg $\newline
$ V = \frac{4\pi r^3}{3} = \frac{4\pi 0,0125^3}{3} \approx 8,181230869*10^{-6} m^3 $\newline
$ \rho = \frac{m}{V} = \frac{0,061}{8,181230869*10^{-6}} \approx 7,5*10^3 kg/m^3 $\newline
\newline
EX2:\newline
Hur mycket korv kan man göra av Thomas?\newline
$ V = A*l $\newline
Thomas volym?\newline
Thomas massa: $ m=110kg $\newline
$ V \rho = \frac{mV}{\rho} $\newline
$ \frac{V\rho}{\rho} = \frac{m}{\rho} $\newline
$ V = \frac{m}{\rho} $ \newline
Thomas densitet $ \approx $ vattnets densitet.\newline
$ \rho = 0,998 g/cm^3 = 998 kg/m^3 $\newline
$ V= \frac{m}{\rho} = 0,11 m^3 $\newline
$ r = 1,5 cm $ Thomas korv\newline
$ A = r^2 \pi = (0,015)^2 =\approx 7,068*10^-4 $\newline
$ \rho = \frac{V}{A} = \frac{0,11}{7,068*10^-4} $\newline
\newpage
EX3:\newline
Uppskatta massan för luften i föreläsnings salen.\newline
$ \rho = \frac{m V}{V} $\newline
$ m = \rho V = 1293 * 540 \approx 700 kg $\newline
$ \rho = 1,293 kg/m^3 $\newline
$ V = 12 * 15 * 3  \approx 540 m^3 $
\newpage
Mätnoggranhet\newline
Anger närmevärdet med felgränsen\newline
$ A = 0,305 m^2 $\newline
$ 0,3045 \leqslant  A \leqslant  0,3055 m^3 $ 3 gällande siffror\newline
\newpage
Viktig regel\newline
Om du gör en multiplikation eller division ska svaret vara så många gällande siffror som det minst noggranna ingångs värde\newline
EX1:\newline
En matta har längden(\textit{l}) 12,71 m och bredden(\textit{b}) 3,46 m.\newline
Vilken area har mattan?\newline
$ A = lb = 12,71 * 3,46 \approx 43,9766 m^2 \approx 44,0 m^2 $\newline
Om du gör en addition eller subtraktion ska svaret ha lika många decimaler som det ingångsvärde som har minst antal decimaler.
\section{Övningar}
\subsection{Densitet}
Koppar folie massa: $m=13g=0,013kg $\newline
Koppar folie densitet: $ \rho=\frac{m}{V} $ $ V=\frac{m}{\rho} = \frac{0,013}{8,96*10^3} $\newline
$ h=\frac{V}{A}=1,45*10^{-6} $
\subsection{Mätning}
$ t=\frac{13min}{2}=6,5min $
$ v= 0,300*10^4 m/s $
$ v=\frac{s}{t} $
$ s=v*t = (0,300*10^9)*(6,5*60)=1,2*10^{11}m $
\end{flushleft}
\end{document}
